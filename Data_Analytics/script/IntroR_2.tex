% Options for packages loaded elsewhere
\PassOptionsToPackage{unicode}{hyperref}
\PassOptionsToPackage{hyphens}{url}
%
\documentclass[
]{article}
\usepackage{amsmath,amssymb}
\usepackage{iftex}
\ifPDFTeX
  \usepackage[T1]{fontenc}
  \usepackage[utf8]{inputenc}
  \usepackage{textcomp} % provide euro and other symbols
\else % if luatex or xetex
  \usepackage{unicode-math} % this also loads fontspec
  \defaultfontfeatures{Scale=MatchLowercase}
  \defaultfontfeatures[\rmfamily]{Ligatures=TeX,Scale=1}
\fi
\usepackage{lmodern}
\ifPDFTeX\else
  % xetex/luatex font selection
\fi
% Use upquote if available, for straight quotes in verbatim environments
\IfFileExists{upquote.sty}{\usepackage{upquote}}{}
\IfFileExists{microtype.sty}{% use microtype if available
  \usepackage[]{microtype}
  \UseMicrotypeSet[protrusion]{basicmath} % disable protrusion for tt fonts
}{}
\makeatletter
\@ifundefined{KOMAClassName}{% if non-KOMA class
  \IfFileExists{parskip.sty}{%
    \usepackage{parskip}
  }{% else
    \setlength{\parindent}{0pt}
    \setlength{\parskip}{6pt plus 2pt minus 1pt}}
}{% if KOMA class
  \KOMAoptions{parskip=half}}
\makeatother
\usepackage{xcolor}
\usepackage[margin=1in]{geometry}
\usepackage{color}
\usepackage{fancyvrb}
\newcommand{\VerbBar}{|}
\newcommand{\VERB}{\Verb[commandchars=\\\{\}]}
\DefineVerbatimEnvironment{Highlighting}{Verbatim}{commandchars=\\\{\}}
% Add ',fontsize=\small' for more characters per line
\usepackage{framed}
\definecolor{shadecolor}{RGB}{248,248,248}
\newenvironment{Shaded}{\begin{snugshade}}{\end{snugshade}}
\newcommand{\AlertTok}[1]{\textcolor[rgb]{0.94,0.16,0.16}{#1}}
\newcommand{\AnnotationTok}[1]{\textcolor[rgb]{0.56,0.35,0.01}{\textbf{\textit{#1}}}}
\newcommand{\AttributeTok}[1]{\textcolor[rgb]{0.13,0.29,0.53}{#1}}
\newcommand{\BaseNTok}[1]{\textcolor[rgb]{0.00,0.00,0.81}{#1}}
\newcommand{\BuiltInTok}[1]{#1}
\newcommand{\CharTok}[1]{\textcolor[rgb]{0.31,0.60,0.02}{#1}}
\newcommand{\CommentTok}[1]{\textcolor[rgb]{0.56,0.35,0.01}{\textit{#1}}}
\newcommand{\CommentVarTok}[1]{\textcolor[rgb]{0.56,0.35,0.01}{\textbf{\textit{#1}}}}
\newcommand{\ConstantTok}[1]{\textcolor[rgb]{0.56,0.35,0.01}{#1}}
\newcommand{\ControlFlowTok}[1]{\textcolor[rgb]{0.13,0.29,0.53}{\textbf{#1}}}
\newcommand{\DataTypeTok}[1]{\textcolor[rgb]{0.13,0.29,0.53}{#1}}
\newcommand{\DecValTok}[1]{\textcolor[rgb]{0.00,0.00,0.81}{#1}}
\newcommand{\DocumentationTok}[1]{\textcolor[rgb]{0.56,0.35,0.01}{\textbf{\textit{#1}}}}
\newcommand{\ErrorTok}[1]{\textcolor[rgb]{0.64,0.00,0.00}{\textbf{#1}}}
\newcommand{\ExtensionTok}[1]{#1}
\newcommand{\FloatTok}[1]{\textcolor[rgb]{0.00,0.00,0.81}{#1}}
\newcommand{\FunctionTok}[1]{\textcolor[rgb]{0.13,0.29,0.53}{\textbf{#1}}}
\newcommand{\ImportTok}[1]{#1}
\newcommand{\InformationTok}[1]{\textcolor[rgb]{0.56,0.35,0.01}{\textbf{\textit{#1}}}}
\newcommand{\KeywordTok}[1]{\textcolor[rgb]{0.13,0.29,0.53}{\textbf{#1}}}
\newcommand{\NormalTok}[1]{#1}
\newcommand{\OperatorTok}[1]{\textcolor[rgb]{0.81,0.36,0.00}{\textbf{#1}}}
\newcommand{\OtherTok}[1]{\textcolor[rgb]{0.56,0.35,0.01}{#1}}
\newcommand{\PreprocessorTok}[1]{\textcolor[rgb]{0.56,0.35,0.01}{\textit{#1}}}
\newcommand{\RegionMarkerTok}[1]{#1}
\newcommand{\SpecialCharTok}[1]{\textcolor[rgb]{0.81,0.36,0.00}{\textbf{#1}}}
\newcommand{\SpecialStringTok}[1]{\textcolor[rgb]{0.31,0.60,0.02}{#1}}
\newcommand{\StringTok}[1]{\textcolor[rgb]{0.31,0.60,0.02}{#1}}
\newcommand{\VariableTok}[1]{\textcolor[rgb]{0.00,0.00,0.00}{#1}}
\newcommand{\VerbatimStringTok}[1]{\textcolor[rgb]{0.31,0.60,0.02}{#1}}
\newcommand{\WarningTok}[1]{\textcolor[rgb]{0.56,0.35,0.01}{\textbf{\textit{#1}}}}
\usepackage{graphicx}
\makeatletter
\def\maxwidth{\ifdim\Gin@nat@width>\linewidth\linewidth\else\Gin@nat@width\fi}
\def\maxheight{\ifdim\Gin@nat@height>\textheight\textheight\else\Gin@nat@height\fi}
\makeatother
% Scale images if necessary, so that they will not overflow the page
% margins by default, and it is still possible to overwrite the defaults
% using explicit options in \includegraphics[width, height, ...]{}
\setkeys{Gin}{width=\maxwidth,height=\maxheight,keepaspectratio}
% Set default figure placement to htbp
\makeatletter
\def\fps@figure{htbp}
\makeatother
\setlength{\emergencystretch}{3em} % prevent overfull lines
\providecommand{\tightlist}{%
  \setlength{\itemsep}{0pt}\setlength{\parskip}{0pt}}
\setcounter{secnumdepth}{-\maxdimen} % remove section numbering
\ifLuaTeX
  \usepackage{selnolig}  % disable illegal ligatures
\fi
\usepackage{bookmark}
\IfFileExists{xurl.sty}{\usepackage{xurl}}{} % add URL line breaks if available
\urlstyle{same}
\hypersetup{
  pdftitle={IntroR\_2.R},
  pdfauthor={frame},
  hidelinks,
  pdfcreator={LaTeX via pandoc}}

\title{IntroR\_2.R}
\author{frame}
\date{2025-03-12}

\begin{document}
\maketitle

\begin{Shaded}
\begin{Highlighting}[]
\CommentTok{\# Tipi di dati e strutture {-}{-}{-}{-}{-}}
\DocumentationTok{\#\# tipi di dati: character, numeric, integer, logical, complex}
\DocumentationTok{\#\# strutture: vector, factor, matrix, array, list, data frame}

\NormalTok{b }\OtherTok{\textless{}{-}}\DecValTok{1}
\CommentTok{\#?str}
\FunctionTok{str}\NormalTok{(b)}
\end{Highlighting}
\end{Shaded}

\begin{verbatim}
##  num 1
\end{verbatim}

\begin{Shaded}
\begin{Highlighting}[]
\DocumentationTok{\#\# Scalari {-}{-}{-}{-}}
\DocumentationTok{\#\#\# numeric {-}{-}{-}{-}}
\NormalTok{a }\OtherTok{\textless{}{-}} \DecValTok{10}
\NormalTok{b }\OtherTok{\textless{}{-}} \DecValTok{3} \SpecialCharTok{/} \DecValTok{10}
\NormalTok{c }\OtherTok{\textless{}{-}}\NormalTok{ (a }\SpecialCharTok{+}\NormalTok{ b) }\SpecialCharTok{/}\NormalTok{ b}
\NormalTok{c}
\end{Highlighting}
\end{Shaded}

\begin{verbatim}
## [1] 34.33333
\end{verbatim}

\begin{Shaded}
\begin{Highlighting}[]
\FunctionTok{str}\NormalTok{(b)}
\end{Highlighting}
\end{Shaded}

\begin{verbatim}
##  num 0.3
\end{verbatim}

\begin{Shaded}
\begin{Highlighting}[]
\DocumentationTok{\#\#\# character {-}{-}{-}{-}}
\NormalTok{d }\OtherTok{\textless{}{-}} \StringTok{"1"}
\NormalTok{e }\OtherTok{\textless{}{-}} \StringTok{"2"}
\CommentTok{\#d+e }

\NormalTok{s }\OtherTok{\textless{}{-}} \StringTok{"hello"}
\FunctionTok{str}\NormalTok{(s)}
\end{Highlighting}
\end{Shaded}

\begin{verbatim}
##  chr "hello"
\end{verbatim}

\begin{Shaded}
\begin{Highlighting}[]
\DocumentationTok{\#\#\# integer {-}{-}{-}{-}}
\NormalTok{var }\OtherTok{\textless{}{-}} \DecValTok{2}
\NormalTok{var.int }\OtherTok{\textless{}{-}} \DecValTok{2}\DataTypeTok{L}

\FunctionTok{str}\NormalTok{(var)}
\end{Highlighting}
\end{Shaded}

\begin{verbatim}
##  num 2
\end{verbatim}

\begin{Shaded}
\begin{Highlighting}[]
\FunctionTok{str}\NormalTok{(var.int)}
\end{Highlighting}
\end{Shaded}

\begin{verbatim}
##  int 2
\end{verbatim}

\begin{Shaded}
\begin{Highlighting}[]
\DocumentationTok{\#\#\# logical {-}{-}{-}{-}}

\NormalTok{x }\OtherTok{\textless{}{-}} \ConstantTok{TRUE}
\NormalTok{y }\OtherTok{\textless{}{-}} \ConstantTok{FALSE}

\NormalTok{x}\SpecialCharTok{+}\NormalTok{y}
\end{Highlighting}
\end{Shaded}

\begin{verbatim}
## [1] 1
\end{verbatim}

\begin{Shaded}
\begin{Highlighting}[]
\DecValTok{2}\SpecialCharTok{*}\NormalTok{x}
\end{Highlighting}
\end{Shaded}

\begin{verbatim}
## [1] 2
\end{verbatim}

\begin{Shaded}
\begin{Highlighting}[]
\FunctionTok{str}\NormalTok{(x)}
\end{Highlighting}
\end{Shaded}

\begin{verbatim}
##  logi TRUE
\end{verbatim}

\begin{Shaded}
\begin{Highlighting}[]
\DocumentationTok{\#\#\#\# verificare o convertire il tipo di oggetto {-}{-}{-}{-}}

\CommentTok{\# is.numeric/character/integer/logical...}
\FunctionTok{is.integer}\NormalTok{(var)}
\end{Highlighting}
\end{Shaded}

\begin{verbatim}
## [1] FALSE
\end{verbatim}

\begin{Shaded}
\begin{Highlighting}[]
\FunctionTok{is.integer}\NormalTok{(var.int)}
\end{Highlighting}
\end{Shaded}

\begin{verbatim}
## [1] TRUE
\end{verbatim}

\begin{Shaded}
\begin{Highlighting}[]
\CommentTok{\#is.numeric()}

\CommentTok{\# as.numeric/character/integer/logical...}
\CommentTok{\#as.numeric(s) \#s character}
\FunctionTok{as.integer}\NormalTok{(b) }\CommentTok{\#b numeric}
\end{Highlighting}
\end{Shaded}

\begin{verbatim}
## [1] 0
\end{verbatim}

\begin{Shaded}
\begin{Highlighting}[]
\FunctionTok{as.numeric}\NormalTok{(}\ConstantTok{FALSE}\NormalTok{) }\CommentTok{\#valori logici}
\end{Highlighting}
\end{Shaded}

\begin{verbatim}
## [1] 0
\end{verbatim}

\begin{Shaded}
\begin{Highlighting}[]
\FunctionTok{as.character}\NormalTok{(a)}
\end{Highlighting}
\end{Shaded}

\begin{verbatim}
## [1] "10"
\end{verbatim}

\begin{Shaded}
\begin{Highlighting}[]
\DocumentationTok{\#\# Vettori {-}{-}{-}{-}}
\CommentTok{\# un vettore ammette solo elementi dello stesso tipo}
\CommentTok{\# c(); vector()}
\NormalTok{db\_v }\OtherTok{\textless{}{-}} \FunctionTok{c}\NormalTok{(}\DecValTok{10}\NormalTok{, }\DecValTok{15}\NormalTok{, }\FloatTok{6.4}\NormalTok{, }\DecValTok{3}\NormalTok{, }\DecValTok{18}\NormalTok{) }\CommentTok{\# definiamo il vettore x}
\NormalTok{db\_v                          }\CommentTok{\# stampiamo x}
\end{Highlighting}
\end{Shaded}

\begin{verbatim}
## [1] 10.0 15.0  6.4  3.0 18.0
\end{verbatim}

\begin{Shaded}
\begin{Highlighting}[]
\FunctionTok{str}\NormalTok{(db\_v)                     }\CommentTok{\# verifichiamo la struttura di x }
\end{Highlighting}
\end{Shaded}

\begin{verbatim}
##  num [1:5] 10 15 6.4 3 18
\end{verbatim}

\begin{Shaded}
\begin{Highlighting}[]
\NormalTok{int\_v }\OtherTok{\textless{}{-}} \FunctionTok{c}\NormalTok{(}\DecValTok{1}\DataTypeTok{L}\NormalTok{, }\DecValTok{6}\DataTypeTok{L}\NormalTok{, }\DecValTok{10}\DataTypeTok{L}\NormalTok{)}
\NormalTok{int\_v}
\end{Highlighting}
\end{Shaded}

\begin{verbatim}
## [1]  1  6 10
\end{verbatim}

\begin{Shaded}
\begin{Highlighting}[]
\FunctionTok{str}\NormalTok{(int\_v)}
\end{Highlighting}
\end{Shaded}

\begin{verbatim}
##  int [1:3] 1 6 10
\end{verbatim}

\begin{Shaded}
\begin{Highlighting}[]
\NormalTok{log\_v }\OtherTok{\textless{}{-}} \FunctionTok{c}\NormalTok{(}\ConstantTok{TRUE}\NormalTok{, }\ConstantTok{FALSE}\NormalTok{, T, F)}
\NormalTok{log\_v}
\end{Highlighting}
\end{Shaded}

\begin{verbatim}
## [1]  TRUE FALSE  TRUE FALSE
\end{verbatim}

\begin{Shaded}
\begin{Highlighting}[]
\FunctionTok{str}\NormalTok{(log\_v)}
\end{Highlighting}
\end{Shaded}

\begin{verbatim}
##  logi [1:4] TRUE FALSE TRUE FALSE
\end{verbatim}

\begin{Shaded}
\begin{Highlighting}[]
\NormalTok{ch\_v }\OtherTok{\textless{}{-}} \FunctionTok{c}\NormalTok{(}\StringTok{"A"}\NormalTok{, }\StringTok{"B"}\NormalTok{, }\StringTok{"C"}\NormalTok{)}
\NormalTok{ch\_v}
\end{Highlighting}
\end{Shaded}

\begin{verbatim}
## [1] "A" "B" "C"
\end{verbatim}

\begin{Shaded}
\begin{Highlighting}[]
\FunctionTok{str}\NormalTok{(ch\_v)}
\end{Highlighting}
\end{Shaded}

\begin{verbatim}
##  chr [1:3] "A" "B" "C"
\end{verbatim}

\begin{Shaded}
\begin{Highlighting}[]
\FunctionTok{is.logical}\NormalTok{(log\_v)}
\end{Highlighting}
\end{Shaded}

\begin{verbatim}
## [1] TRUE
\end{verbatim}

\begin{Shaded}
\begin{Highlighting}[]
\FunctionTok{is.numeric}\NormalTok{(log\_v)}
\end{Highlighting}
\end{Shaded}

\begin{verbatim}
## [1] FALSE
\end{verbatim}

\begin{Shaded}
\begin{Highlighting}[]
\DocumentationTok{\#\#\# Creare una sequenza {-}{-}{-}{-}}
\NormalTok{v }\OtherTok{\textless{}{-}} \DecValTok{1}\SpecialCharTok{:}\DecValTok{10}
\NormalTok{v}
\end{Highlighting}
\end{Shaded}

\begin{verbatim}
##  [1]  1  2  3  4  5  6  7  8  9 10
\end{verbatim}

\begin{Shaded}
\begin{Highlighting}[]
\FunctionTok{seq}\NormalTok{(}\AttributeTok{from=}\DecValTok{1}\NormalTok{, }\AttributeTok{to=}\DecValTok{10}\NormalTok{)}
\end{Highlighting}
\end{Shaded}

\begin{verbatim}
##  [1]  1  2  3  4  5  6  7  8  9 10
\end{verbatim}

\begin{Shaded}
\begin{Highlighting}[]
\FunctionTok{seq}\NormalTok{(}\AttributeTok{from=}\DecValTok{1}\NormalTok{, }\AttributeTok{to=}\DecValTok{10}\NormalTok{, }\AttributeTok{by=}\DecValTok{3}\NormalTok{)}
\end{Highlighting}
\end{Shaded}

\begin{verbatim}
## [1]  1  4  7 10
\end{verbatim}

\begin{Shaded}
\begin{Highlighting}[]
\FunctionTok{seq}\NormalTok{(}\AttributeTok{from=}\DecValTok{1}\NormalTok{, }\AttributeTok{to=}\DecValTok{10}\NormalTok{, }\AttributeTok{length=}\DecValTok{5}\NormalTok{)}
\end{Highlighting}
\end{Shaded}

\begin{verbatim}
## [1]  1.00  3.25  5.50  7.75 10.00
\end{verbatim}

\begin{Shaded}
\begin{Highlighting}[]
\FunctionTok{rep}\NormalTok{(}\DecValTok{1}\NormalTok{,}\DecValTok{15}\NormalTok{) }\CommentTok{\#un vettore di lunghezza 15 con tutti gli elementi uguali a 1}
\end{Highlighting}
\end{Shaded}

\begin{verbatim}
##  [1] 1 1 1 1 1 1 1 1 1 1 1 1 1 1 1
\end{verbatim}

\begin{Shaded}
\begin{Highlighting}[]
\NormalTok{a }\OtherTok{\textless{}{-}} \FunctionTok{c}\NormalTok{(}\FunctionTok{rep}\NormalTok{(}\DecValTok{2}\NormalTok{,}\DecValTok{3}\NormalTok{),}\DecValTok{4}\NormalTok{,}\DecValTok{5}\NormalTok{,}\FunctionTok{rep}\NormalTok{(}\DecValTok{1}\NormalTok{,}\DecValTok{5}\NormalTok{),}\DecValTok{11}\SpecialCharTok{:}\DecValTok{15}\NormalTok{)}
\NormalTok{a}
\end{Highlighting}
\end{Shaded}

\begin{verbatim}
##  [1]  2  2  2  4  5  1  1  1  1  1 11 12 13 14 15
\end{verbatim}

\begin{Shaded}
\begin{Highlighting}[]
\FunctionTok{rep}\NormalTok{(}\FunctionTok{c}\NormalTok{(}\DecValTok{1}\NormalTok{, }\DecValTok{2}\NormalTok{), }\DecValTok{3}\NormalTok{)}
\end{Highlighting}
\end{Shaded}

\begin{verbatim}
## [1] 1 2 1 2 1 2
\end{verbatim}

\begin{Shaded}
\begin{Highlighting}[]
\FunctionTok{rep}\NormalTok{(}\FunctionTok{c}\NormalTok{(}\DecValTok{1}\NormalTok{, }\DecValTok{2}\NormalTok{), }\AttributeTok{each=}\DecValTok{3}\NormalTok{)}
\end{Highlighting}
\end{Shaded}

\begin{verbatim}
## [1] 1 1 1 2 2 2
\end{verbatim}

\begin{Shaded}
\begin{Highlighting}[]
\FunctionTok{rep}\NormalTok{(}\FunctionTok{c}\NormalTok{(}\StringTok{"a"}\NormalTok{,}\StringTok{"b"}\NormalTok{), }\DecValTok{5}\NormalTok{)}
\end{Highlighting}
\end{Shaded}

\begin{verbatim}
##  [1] "a" "b" "a" "b" "a" "b" "a" "b" "a" "b"
\end{verbatim}

\begin{Shaded}
\begin{Highlighting}[]
\FunctionTok{rep}\NormalTok{(}\FunctionTok{c}\NormalTok{(}\StringTok{"a"}\NormalTok{,}\StringTok{"b"}\NormalTok{), }\AttributeTok{each =} \DecValTok{5}\NormalTok{)}
\end{Highlighting}
\end{Shaded}

\begin{verbatim}
##  [1] "a" "a" "a" "a" "a" "b" "b" "b" "b" "b"
\end{verbatim}

\begin{Shaded}
\begin{Highlighting}[]
\FunctionTok{rep}\NormalTok{(}\FunctionTok{c}\NormalTok{(a,}\StringTok{"b"}\NormalTok{), }\DecValTok{2}\NormalTok{)          }\CommentTok{\#il vettore è convertito in character}
\end{Highlighting}
\end{Shaded}

\begin{verbatim}
##  [1] "2"  "2"  "2"  "4"  "5"  "1"  "1"  "1"  "1"  "1"  "11" "12" "13" "14" "15"
## [16] "b"  "2"  "2"  "2"  "4"  "5"  "1"  "1"  "1"  "1"  "1"  "11" "12" "13" "14"
## [31] "15" "b"
\end{verbatim}

\begin{Shaded}
\begin{Highlighting}[]
\FunctionTok{rep}\NormalTok{(}\FunctionTok{c}\NormalTok{(a,}\StringTok{"b"}\NormalTok{), }\AttributeTok{each=}\DecValTok{2}\NormalTok{) }
\end{Highlighting}
\end{Shaded}

\begin{verbatim}
##  [1] "2"  "2"  "2"  "2"  "2"  "2"  "4"  "4"  "5"  "5"  "1"  "1"  "1"  "1"  "1" 
## [16] "1"  "1"  "1"  "1"  "1"  "11" "11" "12" "12" "13" "13" "14" "14" "15" "15"
## [31] "b"  "b"
\end{verbatim}

\begin{Shaded}
\begin{Highlighting}[]
\FunctionTok{paste}\NormalTok{(}\StringTok{"file"}\NormalTok{, }\DecValTok{1}\SpecialCharTok{:}\DecValTok{5}\NormalTok{, }\StringTok{".txt"}\NormalTok{ , }\AttributeTok{sep=}\StringTok{""}\NormalTok{)}
\end{Highlighting}
\end{Shaded}

\begin{verbatim}
## [1] "file1.txt" "file2.txt" "file3.txt" "file4.txt" "file5.txt"
\end{verbatim}

\begin{Shaded}
\begin{Highlighting}[]
\FunctionTok{paste}\NormalTok{(}\StringTok{"file"}\NormalTok{, }\DecValTok{1}\NormalTok{, }\StringTok{".txt"}\NormalTok{ , }\AttributeTok{sep=}\StringTok{""}\NormalTok{)}
\end{Highlighting}
\end{Shaded}

\begin{verbatim}
## [1] "file1.txt"
\end{verbatim}

\begin{Shaded}
\begin{Highlighting}[]
\DocumentationTok{\#\# Selezione di elementi ed operazioni}
\CommentTok{\#selezione di elementi (la prima posizione ha indice 1)}
\NormalTok{y }\OtherTok{\textless{}{-}}\NormalTok{ a[}\DecValTok{6}\NormalTok{]        }
\NormalTok{y}
\end{Highlighting}
\end{Shaded}

\begin{verbatim}
## [1] 1
\end{verbatim}

\begin{Shaded}
\begin{Highlighting}[]
\NormalTok{z }\OtherTok{\textless{}{-}}\NormalTok{ a[}\DecValTok{2}\SpecialCharTok{:}\DecValTok{4}\NormalTok{]      }
\NormalTok{z}
\end{Highlighting}
\end{Shaded}

\begin{verbatim}
## [1] 2 2 4
\end{verbatim}

\begin{Shaded}
\begin{Highlighting}[]
\NormalTok{w }\OtherTok{\textless{}{-}}\NormalTok{ a[}\FunctionTok{c}\NormalTok{(}\DecValTok{2}\NormalTok{, }\DecValTok{5}\NormalTok{)] }\CommentTok{\# estrarre l\textquotesingle{}elemento 2 e 5}
\NormalTok{w}
\end{Highlighting}
\end{Shaded}

\begin{verbatim}
## [1] 2 5
\end{verbatim}

\begin{Shaded}
\begin{Highlighting}[]
\NormalTok{x }\OtherTok{\textless{}{-}} \FunctionTok{c}\NormalTok{(}\DecValTok{1}\NormalTok{, }\DecValTok{2}\NormalTok{, }\DecValTok{4}\NormalTok{, }\DecValTok{8}\NormalTok{, }\DecValTok{16}\NormalTok{, }\DecValTok{32}\NormalTok{)}
\NormalTok{x}
\end{Highlighting}
\end{Shaded}

\begin{verbatim}
## [1]  1  2  4  8 16 32
\end{verbatim}

\begin{Shaded}
\begin{Highlighting}[]
\NormalTok{x[}\SpecialCharTok{{-}}\DecValTok{4}\NormalTok{] }\CommentTok{\#!! non è assegnato ad un oggetto, devo fare}
\end{Highlighting}
\end{Shaded}

\begin{verbatim}
## [1]  1  2  4 16 32
\end{verbatim}

\begin{Shaded}
\begin{Highlighting}[]
\NormalTok{z }\OtherTok{\textless{}{-}}\NormalTok{ x[}\SpecialCharTok{{-}}\DecValTok{4}\NormalTok{]}
\NormalTok{z}
\end{Highlighting}
\end{Shaded}

\begin{verbatim}
## [1]  1  2  4 16 32
\end{verbatim}

\begin{Shaded}
\begin{Highlighting}[]
\NormalTok{z }\OtherTok{\textless{}{-}}\NormalTok{ x[}\SpecialCharTok{{-}}\FunctionTok{c}\NormalTok{(}\DecValTok{4}\NormalTok{, }\DecValTok{6}\NormalTok{)] }\CommentTok{\#tolgo due elementi}
\NormalTok{z}
\end{Highlighting}
\end{Shaded}

\begin{verbatim}
## [1]  1  2  4 16
\end{verbatim}

\begin{Shaded}
\begin{Highlighting}[]
\CommentTok{\# operazioni tra scalare e vettore}
\NormalTok{x }\OtherTok{\textless{}{-}} \DecValTok{1}\SpecialCharTok{:}\DecValTok{10}
\NormalTok{x}\SpecialCharTok{*}\DecValTok{2}   
\end{Highlighting}
\end{Shaded}

\begin{verbatim}
##  [1]  2  4  6  8 10 12 14 16 18 20
\end{verbatim}

\begin{Shaded}
\begin{Highlighting}[]
\NormalTok{x }\SpecialCharTok{\textgreater{}} \DecValTok{5} 
\end{Highlighting}
\end{Shaded}

\begin{verbatim}
##  [1] FALSE FALSE FALSE FALSE FALSE  TRUE  TRUE  TRUE  TRUE  TRUE
\end{verbatim}

\begin{Shaded}
\begin{Highlighting}[]
\DocumentationTok{\#\# operazioni tra 2 vettori di lunghezza diversa}
\NormalTok{x }\OtherTok{\textless{}{-}} \FunctionTok{rep}\NormalTok{(}\DecValTok{10}\NormalTok{,}\DecValTok{8}\NormalTok{)}
\NormalTok{y }\OtherTok{\textless{}{-}} \FunctionTok{c}\NormalTok{(}\DecValTok{1}\NormalTok{,}\DecValTok{2}\NormalTok{)}
\NormalTok{x }\OtherTok{\textless{}{-}}\NormalTok{ x[}\SpecialCharTok{{-}}\DecValTok{1}\NormalTok{]}
\NormalTok{y}
\end{Highlighting}
\end{Shaded}

\begin{verbatim}
## [1] 1 2
\end{verbatim}

\begin{Shaded}
\begin{Highlighting}[]
\NormalTok{x}\SpecialCharTok{*}\NormalTok{y   }
\end{Highlighting}
\end{Shaded}

\begin{verbatim}
## Warning in x * y: la lunghezza più lunga dell'oggetto non è un multiplo della
## lunghezza più corta dell'oggetto
\end{verbatim}

\begin{verbatim}
## [1] 10 20 10 20 10 20 10
\end{verbatim}

\begin{Shaded}
\begin{Highlighting}[]
\NormalTok{x}\SpecialCharTok{+}\NormalTok{y}
\end{Highlighting}
\end{Shaded}

\begin{verbatim}
## Warning in x + y: la lunghezza più lunga dell'oggetto non è un multiplo della
## lunghezza più corta dell'oggetto
\end{verbatim}

\begin{verbatim}
## [1] 11 12 11 12 11 12 11
\end{verbatim}

\begin{Shaded}
\begin{Highlighting}[]
\CommentTok{\#operatori logici: OR |, AND \&}
\NormalTok{x }\OtherTok{\textless{}{-}} \DecValTok{1}\SpecialCharTok{:}\DecValTok{8}

\CommentTok{\# selezioniamo gli elementi maggiori o }
\CommentTok{\# uguali a 7 o minori di 5}
\NormalTok{x}
\end{Highlighting}
\end{Shaded}

\begin{verbatim}
## [1] 1 2 3 4 5 6 7 8
\end{verbatim}

\begin{Shaded}
\begin{Highlighting}[]
\NormalTok{x }\SpecialCharTok{\textgreater{}=} \DecValTok{7}
\end{Highlighting}
\end{Shaded}

\begin{verbatim}
## [1] FALSE FALSE FALSE FALSE FALSE FALSE  TRUE  TRUE
\end{verbatim}

\begin{Shaded}
\begin{Highlighting}[]
\NormalTok{x }\SpecialCharTok{\textless{}} \DecValTok{5}
\end{Highlighting}
\end{Shaded}

\begin{verbatim}
## [1]  TRUE  TRUE  TRUE  TRUE FALSE FALSE FALSE FALSE
\end{verbatim}

\begin{Shaded}
\begin{Highlighting}[]
\NormalTok{x }\SpecialCharTok{\textgreater{}=} \DecValTok{7} \SpecialCharTok{|}\NormalTok{ x }\SpecialCharTok{\textless{}} \DecValTok{5}
\end{Highlighting}
\end{Shaded}

\begin{verbatim}
## [1]  TRUE  TRUE  TRUE  TRUE FALSE FALSE  TRUE  TRUE
\end{verbatim}

\begin{Shaded}
\begin{Highlighting}[]
\NormalTok{x[}\FunctionTok{c}\NormalTok{(x }\SpecialCharTok{\textgreater{}=} \DecValTok{7} \SpecialCharTok{|}\NormalTok{ x }\SpecialCharTok{\textless{}} \DecValTok{5}\NormalTok{)] }\CommentTok{\#seleziono solo gli elementi true}
\end{Highlighting}
\end{Shaded}

\begin{verbatim}
## [1] 1 2 3 4 7 8
\end{verbatim}

\begin{Shaded}
\begin{Highlighting}[]
\NormalTok{x }\OtherTok{\textless{}{-}} \DecValTok{0}\SpecialCharTok{:}\DecValTok{7}
\FunctionTok{all}\NormalTok{(x}\SpecialCharTok{\textgreater{}}\DecValTok{0}\NormalTok{) }\CommentTok{\#tutti soddisfano?}
\end{Highlighting}
\end{Shaded}

\begin{verbatim}
## [1] FALSE
\end{verbatim}

\begin{Shaded}
\begin{Highlighting}[]
\FunctionTok{any}\NormalTok{(x}\SpecialCharTok{\textless{}}\DecValTok{0}\NormalTok{) }\CommentTok{\#almeno uno soddisfa?}
\end{Highlighting}
\end{Shaded}

\begin{verbatim}
## [1] FALSE
\end{verbatim}

\begin{Shaded}
\begin{Highlighting}[]
\CommentTok{\# altre funzioni}
\NormalTok{x }\OtherTok{\textless{}{-}} \FunctionTok{c}\NormalTok{(}\ConstantTok{FALSE}\NormalTok{, }\ConstantTok{FALSE}\NormalTok{, }\ConstantTok{TRUE}\NormalTok{)}

\FunctionTok{sum}\NormalTok{(x)}
\end{Highlighting}
\end{Shaded}

\begin{verbatim}
## [1] 1
\end{verbatim}

\begin{Shaded}
\begin{Highlighting}[]
\FunctionTok{mean}\NormalTok{(x)}
\end{Highlighting}
\end{Shaded}

\begin{verbatim}
## [1] 0.3333333
\end{verbatim}

\begin{Shaded}
\begin{Highlighting}[]
\FunctionTok{length}\NormalTok{(x)}
\end{Highlighting}
\end{Shaded}

\begin{verbatim}
## [1] 3
\end{verbatim}

\begin{Shaded}
\begin{Highlighting}[]
\FunctionTok{as.numeric}\NormalTok{(x)}
\end{Highlighting}
\end{Shaded}

\begin{verbatim}
## [1] 0 0 1
\end{verbatim}

\begin{Shaded}
\begin{Highlighting}[]
\NormalTok{x }\OtherTok{\textless{}{-}} \DecValTok{3}\SpecialCharTok{:}\DecValTok{22}

\FunctionTok{sort}\NormalTok{(x, }\AttributeTok{decreasing =}\NormalTok{ T)}
\end{Highlighting}
\end{Shaded}

\begin{verbatim}
##  [1] 22 21 20 19 18 17 16 15 14 13 12 11 10  9  8  7  6  5  4  3
\end{verbatim}

\begin{Shaded}
\begin{Highlighting}[]
\FunctionTok{order}\NormalTok{(x)}
\end{Highlighting}
\end{Shaded}

\begin{verbatim}
##  [1]  1  2  3  4  5  6  7  8  9 10 11 12 13 14 15 16 17 18 19 20
\end{verbatim}

\begin{Shaded}
\begin{Highlighting}[]
\FunctionTok{max}\NormalTok{(x)}
\end{Highlighting}
\end{Shaded}

\begin{verbatim}
## [1] 22
\end{verbatim}

\begin{Shaded}
\begin{Highlighting}[]
\FunctionTok{min}\NormalTok{(x)}
\end{Highlighting}
\end{Shaded}

\begin{verbatim}
## [1] 3
\end{verbatim}

\begin{Shaded}
\begin{Highlighting}[]
\FunctionTok{range}\NormalTok{(x)}
\end{Highlighting}
\end{Shaded}

\begin{verbatim}
## [1]  3 22
\end{verbatim}

\begin{Shaded}
\begin{Highlighting}[]
\FunctionTok{sum}\NormalTok{(x)}
\end{Highlighting}
\end{Shaded}

\begin{verbatim}
## [1] 250
\end{verbatim}

\begin{Shaded}
\begin{Highlighting}[]
\FunctionTok{prod}\NormalTok{(x)}
\end{Highlighting}
\end{Shaded}

\begin{verbatim}
## [1] 5.620004e+20
\end{verbatim}

\begin{Shaded}
\begin{Highlighting}[]
\FunctionTok{mean}\NormalTok{(x)}
\end{Highlighting}
\end{Shaded}

\begin{verbatim}
## [1] 12.5
\end{verbatim}

\begin{Shaded}
\begin{Highlighting}[]
\CommentTok{\# nomi}
\NormalTok{x }\OtherTok{\textless{}{-}} \FunctionTok{c}\NormalTok{(}\AttributeTok{a =} \DecValTok{1}\NormalTok{, }\AttributeTok{b =} \DecValTok{2}\NormalTok{, }\AttributeTok{c =} \DecValTok{3}\NormalTok{)}
\NormalTok{x}
\end{Highlighting}
\end{Shaded}

\begin{verbatim}
## a b c 
## 1 2 3
\end{verbatim}

\begin{Shaded}
\begin{Highlighting}[]
\NormalTok{x[}\StringTok{"a"}\NormalTok{]}
\end{Highlighting}
\end{Shaded}

\begin{verbatim}
## a 
## 1
\end{verbatim}

\begin{Shaded}
\begin{Highlighting}[]
\NormalTok{x }\OtherTok{\textless{}{-}} \DecValTok{1}\SpecialCharTok{:}\DecValTok{3}
\FunctionTok{names}\NormalTok{(x)}
\end{Highlighting}
\end{Shaded}

\begin{verbatim}
## NULL
\end{verbatim}

\begin{Shaded}
\begin{Highlighting}[]
\FunctionTok{names}\NormalTok{(x) }\OtherTok{\textless{}{-}} \FunctionTok{c}\NormalTok{(}\StringTok{"a1"}\NormalTok{,}\StringTok{"b1"}\NormalTok{,}\StringTok{"c1"}\NormalTok{)}
\NormalTok{x}
\end{Highlighting}
\end{Shaded}

\begin{verbatim}
## a1 b1 c1 
##  1  2  3
\end{verbatim}

\begin{Shaded}
\begin{Highlighting}[]
\DocumentationTok{\#\#\#\#\#\#\#\# Esercizio \#\#\#\#\#\#\#\#\#}
\CommentTok{\#1. Definisci il vettore y con gli elementi 8, 3, 5, 7, 6, 6, 8, 9, 2.}
\CommentTok{\# Gli elementi di y sono minori di 5? }
\CommentTok{\# Crea un nuovo vettore z con gli elementi di y minori di 5.}

\NormalTok{y }\OtherTok{\textless{}{-}} \FunctionTok{c}\NormalTok{(}\DecValTok{8}\NormalTok{, }\DecValTok{3}\NormalTok{, }\DecValTok{5}\NormalTok{, }\DecValTok{7}\NormalTok{, }\DecValTok{6}\NormalTok{, }\DecValTok{6}\NormalTok{, }\DecValTok{8}\NormalTok{, }\DecValTok{9}\NormalTok{, }\DecValTok{2}\NormalTok{)}
\NormalTok{z }\OtherTok{\textless{}{-}}\NormalTok{ y[}\FunctionTok{c}\NormalTok{(y}\SpecialCharTok{\textless{}}\DecValTok{5}\NormalTok{)]}


\CommentTok{\#2. Fornisci un esempio in cui valori logici sono convertiti in numerici 0{-}1}
\CommentTok{\# utilizzando un operatore aritmetico}

\NormalTok{x }\OtherTok{\textless{}{-}} \FunctionTok{c}\NormalTok{(F, T, T, F)}
\NormalTok{y }\OtherTok{\textless{}{-}} \FunctionTok{as.numeric}\NormalTok{(x) }\CommentTok{\#così senza operatore aritmetico}
\NormalTok{z }\OtherTok{\textless{}{-}}\NormalTok{ x}\SpecialCharTok{*}\DecValTok{1}

\FunctionTok{set.seed}\NormalTok{(}\DecValTok{123}\NormalTok{) }\CommentTok{\#così si possono ottenere numeri "casuali" uguali agli altri}
\FunctionTok{runif}\NormalTok{(}\DecValTok{3}\NormalTok{)}
\end{Highlighting}
\end{Shaded}

\begin{verbatim}
## [1] 0.2875775 0.7883051 0.4089769
\end{verbatim}

\begin{Shaded}
\begin{Highlighting}[]
\CommentTok{\#3. Crea un vettore logico di lunghezza 3. Quindi, moltiplica questo vettore}
\CommentTok{\# tramite runif(3). Cosa succede?}

\NormalTok{x }\OtherTok{\textless{}{-}} \FunctionTok{c}\NormalTok{(T, F, F)}
\NormalTok{x}\SpecialCharTok{*}\FunctionTok{runif}\NormalTok{(}\DecValTok{3}\NormalTok{)}
\end{Highlighting}
\end{Shaded}

\begin{verbatim}
## [1] 0.8830174 0.0000000 0.0000000
\end{verbatim}

\begin{Shaded}
\begin{Highlighting}[]
\CommentTok{\#4.}
\FunctionTok{set.seed}\NormalTok{(}\DecValTok{123}\NormalTok{)}
\NormalTok{x }\OtherTok{\textless{}{-}} \FunctionTok{sample}\NormalTok{(}\DecValTok{10}\NormalTok{) }\SpecialCharTok{\textless{}} \DecValTok{4}
\NormalTok{x}
\end{Highlighting}
\end{Shaded}

\begin{verbatim}
##  [1]  TRUE FALSE  TRUE FALSE FALSE FALSE  TRUE FALSE FALSE FALSE
\end{verbatim}

\begin{Shaded}
\begin{Highlighting}[]
\FunctionTok{which}\NormalTok{(x)}
\end{Highlighting}
\end{Shaded}

\begin{verbatim}
## [1] 1 3 7
\end{verbatim}

\begin{Shaded}
\begin{Highlighting}[]
\FunctionTok{which}\NormalTok{(}\FunctionTok{sample}\NormalTok{(}\DecValTok{10}\NormalTok{) }\SpecialCharTok{\textless{}} \DecValTok{4}\NormalTok{)}
\end{Highlighting}
\end{Shaded}

\begin{verbatim}
## [1]  3  5 10
\end{verbatim}

\begin{Shaded}
\begin{Highlighting}[]
\CommentTok{\#   Cosa fa questo codice?}
\CommentTok{\# genera una lista di 10 numeri da 1 a 10 in ordine casuale e dice quali}
\CommentTok{\#indici sono i numeri minori di 4}

\CommentTok{\#5. Definisci un vettore con valori 9, 2, 3, 9, 4, 10, 4, 11. Scrivi il}
\CommentTok{\# codice per calcolare la somma dei tre valori più grandi (Suggerimento: usa}
\CommentTok{\# la funzione rev).}

\NormalTok{x }\OtherTok{\textless{}{-}} \FunctionTok{c}\NormalTok{(}\DecValTok{9}\NormalTok{, }\DecValTok{2}\NormalTok{, }\DecValTok{3}\NormalTok{, }\DecValTok{9}\NormalTok{, }\DecValTok{4}\NormalTok{, }\DecValTok{10}\NormalTok{, }\DecValTok{4}\NormalTok{, }\DecValTok{11}\NormalTok{)}
\NormalTok{y }\OtherTok{\textless{}{-}} \FunctionTok{rev}\NormalTok{(}\FunctionTok{sort}\NormalTok{(x))}
\FunctionTok{sum}\NormalTok{(y[}\DecValTok{1}\SpecialCharTok{:}\DecValTok{3}\NormalTok{])}
\end{Highlighting}
\end{Shaded}

\begin{verbatim}
## [1] 30
\end{verbatim}

\begin{Shaded}
\begin{Highlighting}[]
\CommentTok{\#6. Crea un vettore x di lunghezza 4. Cosa restituisce il codice?}
\NormalTok{x }\OtherTok{\textless{}{-}} \FunctionTok{c}\NormalTok{(}\DecValTok{1}\NormalTok{, }\DecValTok{2}\NormalTok{, }\DecValTok{3}\NormalTok{, }\DecValTok{4}\NormalTok{)}
\NormalTok{x[}\FunctionTok{c}\NormalTok{(}\ConstantTok{TRUE}\NormalTok{, T, }\ConstantTok{NA}\NormalTok{, }\ConstantTok{FALSE}\NormalTok{)]}
\end{Highlighting}
\end{Shaded}

\begin{verbatim}
## [1]  1  2 NA
\end{verbatim}

\begin{Shaded}
\begin{Highlighting}[]
\CommentTok{\#fa vedere solo quelli True,(come prima con la condizione), NA rimane così }
\CommentTok{\#(Not Available)}
\end{Highlighting}
\end{Shaded}


\end{document}
